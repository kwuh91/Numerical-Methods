\documentclass[a4paper, 12pt]{extarticle}

\usepackage[utf8]{inputenc}
\usepackage[paper=a4paper, left=10mm, right=10mm, top=20mm, bottom=20mm]{geometry}
\usepackage[russian]{babel}
\usepackage{amsmath}
\usepackage{amsfonts}
\usepackage[T2A]{fontenc}
\usepackage{textcomp}
\usepackage{parskip}
\usepackage{graphicx}
\usepackage{setspace}

\linespread{1}

\begin{document}

% титульник

\pagestyle{empty}

\section*{\centering{ \large МИНИСТЕРСТВО ОБРАЗОВАНИЯ И НАУКИ РОССИЙСКОЙ ФЕДЕРАЦИИ ГОСУДАРСТВЕННОЕ БЮДЖЕТНОЕ ОБРАЗОВАТЕЛЬНОЕ УЧРЕЖДЕНИЕ ВЫСШЕГО ПРОФЕССИОНАЛЬНОГО ОБРАЗОВАНИЯ}}

\vspace{3\baselineskip}

\section*{\centering{ \large «Московский государственный технический университет имени Н.Э. Баумана» (МГТУ им. Н.Э. Баумана)}}

\vspace{3\baselineskip}

\section*{\centering{ \large ФАКУЛЬТЕТ ФУНДАМЕНТАЛЬНЫЕ НАУКИ }}
\vspace{-1\baselineskip}
\section*{\centering{ \large КАФЕДРА «ВЫЧИСЛИТЕЛЬНАЯ МАТЕМАТИКА И МАТЕМАТИЧЕСКАЯ ФИЗИКА»}}

\vspace{1\baselineskip}

\section*{\centering \normalsize Направление: \fontseries{bx}\selectfont{Математика и компьютерные науки}}

\vspace{1\baselineskip}

\section*{\centering{ \normalsize Дисциплина: Численные методы}}

\vspace{1\baselineskip}

\section*{\centering{ \normalsize Домашняя работа \textnumero 1}}
\vspace{-1\baselineskip}
\section*{\centering{ \normalsize «Погрешности при решении СЛАУ»}}
\vspace{-1\baselineskip}
\section*{\centering{ \normalsize Группа ФН11-52Б}}

\vspace{1\baselineskip}

\section*{\centering{ \normalsize Вариант 9}}

\vspace{1\baselineskip}

\begin{flushright}
    { \normalsize \textbf{Студент: Очкин Н. В. }}
\end{flushright}

\vspace{1\baselineskip}

\begin{flushright}
    { \normalsize \textbf{Преподаватель: Кутыркин В. А.}}
\end{flushright}

\vspace{1\baselineskip}

\begin{flushright}
    { \normalsize \textbf{Оценка: } \hspace{2.5cm} }
\end{flushright}

\vspace{2\baselineskip}

\section*{\centering{ \normalsize Москва 2024}}

\newpage

% Задание 1.1

\pagestyle{plain}
\setcounter{page}{1}

\section*{\centering{ \large Задание 1.1}}

Дана СЛАУ ( $N$ – номер студента в журнале, $\alpha = \frac{n - 50}{100}$, где $n$ – номер группы):

\begin{equation}
    \begin{cases}
        150(1 + 0.5 N + \alpha)x^1 + 150(1 + 0.5 N)x^2 + 150(1 + 0.5N)x^3 = 150(3 + 1.5N + \alpha);\\
        150.1 \cdot (1 + 0.5 N)x^1 + 149.9 \cdot (1 + 0.5 N + \alpha)x^2 + 150(1 + 0.5N)x^3 = 150(3 + 1.5 N + \alpha);\\
        149.9 \cdot (1 + 0.5N)x^1 + 150 \cdot (1 + 0.5 N)x^2 + 150.1 \cdot (1 + 0.5 N + \alpha)x^3 = 150 (3 + 1.5 N + \alpha);
    \end{cases}
\end{equation}

Предполагается, что ошибка в матрице этой СЛАУ достаточно мала и относительная
ошибка в её правой части равна 0,01. Приближённая СЛАУ имеет вид:

\begin{equation}
    \begin{cases}
        150(1 + 0.5 N + \alpha)x^1 + 150(1 + 0.5 N)x^2 + 150(1 + 0.5N)x^3 = 150(3 + 1.5N + \alpha)(1 + 0.01);\\
        150.1 \cdot (1 + 0.5 N)x^1 + 149.9 \cdot (1 + 0.5 N + \alpha)x^2 + 150(1 + 0.5N)x^3 = 150(3 + 1.5 N + \alpha)(1 - 0.01);\\
        149.9 \cdot (1 + 0.5N)x^1 + 150 \cdot (1 + 0.5 N)x^2 + 150.1 \cdot (1 + 0.5 N + \alpha)x^3 = 150 (3 + 1.5 N + \alpha)(1 + 0.01);
    \end{cases}
\end{equation}

Требуется найти число обусловленности матрицы рассматриваемой СЛАУ и
относительную погрешность в решении приближённой СЛАУ. Затем, прокомментировать
получившиеся результаты.

\section*{\centering{ \large Решение}}

Подставим $N = 9$ и $\alpha = \frac{52 - 50}{100} = 0.02$ в (1):

\begin{equation}
    \begin{cases}
        828x^1 + 825x^2 + 825x^3 = 2478;\\
        825.55x^1 + 827.448x^2 + 825x^3 = 2478;\\
        824.45x^1 + 825x^2 + 828.552x^3 = 2478;
    \end{cases}
\end{equation}

Матрица СЛАУ (1) имеет вид:

\begin{equation}
    A = \begin{pmatrix}
        828    & 825     & 825 \\
        825.55 & 827.448 & 825 \\
        824.45 & 825     & 828.552
    \end{pmatrix}
\end{equation}

\textbf{1. Найдем число обусловленности матрицы A.}

Число обусловленности матрицы равно:

\begin{equation}
    \textrm{cond}(A) = \lVert A \rVert \cdot \lVert A^{-1} \rVert
\end{equation}

Вычислим обратную матрицу с помощью функции inv() библиотеки sympy:

\begin{equation}
    A^{-1} = \begin{pmatrix}
        0.230115056512510   & -0.135989231310137 & -0.0937223080651046 \\ 
        -0.178193673016471  & 0.272396398402827  & -0.0937988785782234 \\ 
        -0.0515460443076003 & -0.135912660797018 & 0.187861995036292 
    \end{pmatrix}
\end{equation}

Норма матрицы вычисляется по формуле:
\begin{equation}
    \lVert A \rVert = \textrm{max} \left\{ \sum_{j=1}^{n} |a^i_j|:i = \overline{1, n} \right\}
\end{equation}

% Для этого сначала вычислим матрицы, составленные из модулей элементов исходных
% матриц, воспользовавшись функцией applyfunc(abs):

% \begin{equation}
%     |A| = \begin{pmatrix}
%         828.000000000000 & 825.000000000000 & 825.000000000000 \\ 
%         825.550000000000 & 827.448000000000 & 825.000000000000 \\ 
%         824.450000000000 & 825.000000000000 & 828.552000000000 
%     \end{pmatrix}
% \end{equation}

% \begin{equation}
%     |A^{-1}| = \begin{pmatrix}
%         0.230115056512510 & 0.135989231310137 & 0.0937223080651046 \\ 
%         0.178193673016471 & 0.272396398402827 & 0.0937988785782234 \\ 
%         0.0515460443076003 & 0.135912660797018 & 0.187861995036292 
%     \end{pmatrix}
% \end{equation}

Вычислим нормы:

\begin{equation}
    \lVert A \rVert = 2478.002 
\end{equation}

\begin{equation}
    \lVert A^{-1} \rVert = 0.544388949997522
\end{equation}

Теперь вычислим число обусловленности матрицы:

\begin{equation}
    \textrm{cond}(A) = \lVert A \rVert \cdot \lVert A^{-1} \rVert = 2478.002 \cdot 0.544388949997522 \approx 1348.99690687176
\end{equation}

Следовательно, СЛАУ (3) плохо обусловлена.

\textbf{2. Найдем относительную погрешность в решении приближенной СЛАУ (2).}

Подставим $N = 9$ и $\alpha = \frac{52 - 50}{100} = 0.02$ в (2):

\begin{equation}
    \begin{cases}
        828x^1 + 825x^2 + 825x^3 = 2502.78;\\
        825.55x^1 + 827.448x^2 + 825x^3 = 2453.22;\\
        824.45x^1 + 825x^2 + 828.552x^3 = 2502.78;
    \end{cases}
\end{equation}

Запишем СЛАУ (3) в матричной форме:

\begin{equation}
    \begin{pmatrix}
        828    & 825     & 825 \\ 
        825.55 & 827.448 & 825 \\ 
        824.45 & 825     & 828.552 
    \end{pmatrix}
    \begin{pmatrix}
        x^1 \\ 
        x^2 \\ 
        x^3 
    \end{pmatrix}
    =
    \begin{pmatrix}
        2478 \\ 
        2478 \\ 
        2478 
    \end{pmatrix}
\end{equation}

\begin{equation*}
    ^>x = A^{-1}{^>b} =
\end{equation*}

\begin{equation*}
    = 
    \begin{pmatrix}
        0.230115056512510   & -0.135989231310137 & -0.0937223080651046 \\ 
        -0.178193673016471  & 0.272396398402827  & -0.0937988785782234 \\ 
        -0.0515460443076003 & -0.135912660797018 & 0.187861995036292 
    \end{pmatrix}
    \begin{pmatrix}
        2478 \\ 
        2478 \\ 
        2478 
    \end{pmatrix}
    =
\end{equation*}

\begin{equation}
    =
    \begin{pmatrix}
        0.999915466153453 \\
        1.00073239055399 \\
        0.999352450688377
    \end{pmatrix}
    \approx
    \begin{pmatrix}
        1 \\
        1 \\
        1
    \end{pmatrix}
\end{equation}

Запишем СЛАУ (11) в матричной форме:

\begin{equation}
    \begin{pmatrix}
        828    & 825     & 825 \\ 
        825.55 & 827.448 & 825 \\ 
        824.45 & 825     & 828.552 
    \end{pmatrix}
    \begin{pmatrix}
        x^1 \\ 
        x^2 \\ 
        x^3 
    \end{pmatrix}
    =
    \begin{pmatrix}
        2502.78 \\ 
        2453.22 \\ 
        2502.78 
    \end{pmatrix}
\end{equation}

\begin{equation}
    \Delta b =
    \begin{pmatrix}
        2502.78 \\ 
        2453.22 \\ 
        2502.78 
    \end{pmatrix}
    -
    \begin{pmatrix}
        2478 \\ 
        2478 \\ 
        2478 
    \end{pmatrix}
    =
    \begin{pmatrix}
        24.78 \\
        -24.78 \\
        24.78         
    \end{pmatrix}
\end{equation}

\begin{equation*}
    \textrm{Тогда ошибка } ^> \Delta x = A^{-1}{^> \Delta b} =
\end{equation*}

\begin{equation*}
    = 
    \begin{pmatrix}
        0.230115056512510   & -0.135989231310137 & -0.0937223080651046 \\ 
        -0.178193673016471  & 0.272396398402827  & -0.0937988785782234 \\ 
        -0.0515460443076003 & -0.135912660797018 & 0.187861995036292 
    \end{pmatrix}
    \begin{pmatrix}
        24.78 \\
        -24.78 \\
        24.78         
    \end{pmatrix}
    =
\end{equation*}

\begin{equation}
    =
    \begin{pmatrix}
        6.74962545839196 \\
        -13.4899581809387 \\
        6.74582499360713        
    \end{pmatrix}
\end{equation}

\begin{spacing}{1.5}
Поскольку $\lVert ^>x \rVert = 1$ и $\lVert ^> \Delta x \rVert = 6.74962545839196$, \\
относительная погрешность $\frac{\lVert ^> \Delta x \rVert}{\lVert ^>x \rVert} = 6.74962545839196$
\end{spacing}

\begin{spacing}{1.5}
Действительно, \\
$\frac{\lVert ^> \Delta x \rVert}{\lVert ^>x \rVert} = 6.74962545839196$ $\leq$ cond($A$) $\cdot$  $\frac{\lVert ^> \Delta b \rVert}{\lVert ^>b \rVert} = 1242.47190776588 \cdot 0.01 = 12.4247190776588$
\end{spacing}

\section*{\centering{ \large Результаты}}

\begin{spacing}{1.5}
Число обусловленности cond($A$) = $\lVert A \rVert \cdot \lVert A^{-1} \rVert = 1242.47190776588 > 10^2$, значит, матрица СЛАУ плохо обусловлена. 
Относительная погрешность решения $\frac{\lVert ^> \Delta x \rVert}{\lVert ^>x \rVert} = 6.74962545839196$ очень велика вследствие плохой обусловленности СЛАУ.
\end{spacing}

\section*{\centering{ \large Задание 1.2}}

Исходные данные варианта 9: \\

% $N = 9, \alpha = \frac{55 - 52}{100} = 0.03, \lambda + \alpha = 0.4, \lambda = 0.37, F = \arctan, a = 0, b = 1$  

\renewcommand{\arraystretch}{2} % Increase the vertical spacing between rows by a factor of 2

\begin{table}[h]
    \centering
    \begin{tabular}{|c|c|c|c|c|c|c|}
        \hline
        $N$ & $\alpha$ & $\lambda + \alpha$ & $\lambda$ & $F$ & $a$ & $b$ \\
        \hline
        $9$ & $\frac{55 - 52}{100} = 0.03$ & $0.4$ & $0.37$ & $\arctan$ & $0$ & $1$ \\
        \hline
    \end{tabular}
\end{table}

Согласно этой таблице, на отрезке [a;b] выбрана центрально равномерная сетка с десятью узлами

\begin{align*}
    s_1 = \tau_1 = a + \frac{h}{2}, \\ 
    s_2 = \tau_2 = \tau_1 + h, \\
    s_3 = \tau_3 = \tau_2 + h, \\ 
\end{align*}
\vspace{-3\baselineskip}
\begin{equation*}
    \dots
\end{equation*}
\vspace{-1\baselineskip}
\begin{equation*}
    s_{10} = \tau_{10} = \tau_9 + h, 
\end{equation*}

имеющая шаг $h = \frac{b - a}{10} = 0.1$ \\\\
Требуется решить приближённую СЛАУ:
\begin{equation}
    (E + \lambda A){^>x} = {^>b} + {^> \Delta b}, \textrm{ где}
\end{equation}

$
\lambda = 0.37; \\\\ 
E \in GL(\mathbb{R}; 10) - \textrm{единичная матрица}; \\\\
A = (a^i_j)^{10}_{10} \in GL(\mathbb{R}; 10) - \textrm{ матрица}, 
\textrm{ которая определяется соотношением:} \\ a^i_j = F(s_i \cdot \tau_j) \frac{b - a}{10} \textrm{ для }  i, j = \overline{1, 10};\\\\
{^>b} = [b^1, \dots, b^{10} \rangle \in {^> \mathbb{R} ^ {10}} -  \textrm{ вектор},
\textrm{ который определяется соотношением:} \\ {^>b} = (E + \lambda A) \cdot {^>x_*} \textrm{ и } {^>x_*} = 
[ 1,1, \dots, 1 \rangle \in {^> \mathbb{R} ^ {10}} 
$

\begin{spacing}{1.5}
Найти число обусловленности матрицы рассматриваемой СЛАУ и относительную
погрешность в решении приближённой СЛАУ (17). Затем, прокомментировать
получившиеся результаты. Кроме того, найти решение СЛАУ, которая получается из
СЛАУ (17) делением каждого её $i$ -го уравнения ($i = \overline{1,10}$) на число $b_i + \Delta b_i$ . После этого
сравнить абсолютную погрешность в решении получившейся СЛАУ с абсолютной
погрешностью в решении приближённой СЛАУ (17).
\end{spacing}

\section*{\centering{ \large Решение}}

\begin{equation}
    A = \begin{pmatrix}
        0.00025 & 0.00075 & 0.00125 & 0.00175 & 0.00225 & 0.00275 & 0.00325 & 0.00375 & 0.00425 & 0.00475 \\ 
        0.00075 & 0.00225 & 0.00375 & 0.00525 & 0.00674 & 0.00823 & 0.00972 & 0.0112 & 0.01268 & 0.01415 \\ 
        0.00125 & 0.00375 & 0.00624 & 0.00873 & 0.0112 & 0.01366 & 0.01611 & 0.01853 & 0.02094 & 0.02332 \\ 
        0.00175 & 0.00525 & 0.00873 & 0.01219 & 0.01562 & 0.01902 & 0.02237 & 0.02567 & 0.02892 & 0.0321 \\ 
        0.00225 & 0.00674 & 0.0112 & 0.01562 & 0.01998 & 0.02426 & 0.02846 & 0.03255 & 0.03653 & 0.0404 \\ 
        0.00275 & 0.00823 & 0.01366 & 0.01902 & 0.02426 & 0.02937 & 0.03433 & 0.03912 & 0.04373 & 0.04815 \\ 
        0.00325 & 0.00972 & 0.01611 & 0.02237 & 0.02846 & 0.03433 & 0.03998 & 0.04536 & 0.05048 & 0.05532 \\ 
        0.00375 & 0.0112 & 0.01853 & 0.02567 & 0.03255 & 0.03912 & 0.04536 & 0.05124 & 0.05675 & 0.06191 \\ 
        0.00425 & 0.01268 & 0.02094 & 0.02892 & 0.03653 & 0.04373 & 0.05048 & 0.05675 & 0.06257 & 0.06793 \\ 
        0.00475 & 0.01415 & 0.02332 & 0.0321 & 0.0404 & 0.04815 & 0.05532 & 0.06191 & 0.06793 & 0.07342 
    \end{pmatrix}
\end{equation}

Исходная СЛАУ: 

\begin{equation}
    (E + \lambda A) {^>x} = {^>b}
\end{equation}

\begin{equation}
    E + \lambda A = \begin{pmatrix}
        1.00009 & 0.00028 & 0.00046 & 0.00065 & 0.00083 & 0.00102 & 0.00120 & 0.00139 & 0.00157 & 0.00176 \\ 
        0.00028 & 1.00083 & 0.00139 & 0.00194 & 0.00249 & 0.00305 & 0.00360 & 0.00414 & 0.00469 & 0.00524 \\ 
        0.00046 & 0.00139 & 1.00231 & 0.00323 & 0.00414 & 0.00505 & 0.00596 & 0.00686 & 0.00775 & 0.00863 \\ 
        0.00065 & 0.00194 & 0.00323 & 1.00451 & 0.00578 & 0.00704 & 0.00828 & 0.00950 & 0.01070 & 0.01188 \\ 
        0.00083 & 0.00249 & 0.00414 & 0.00578 & 1.00739 & 0.00898 & 0.01053 & 0.01204 & 0.01352 & 0.01495 \\ 
        0.00102 & 0.00305 & 0.00505 & 0.00704 & 0.00898 & 1.01087 & 0.01270 & 0.01447 & 0.01618 & 0.01782 \\ 
        0.00120 & 0.00360 & 0.00596 & 0.00828 & 0.01053 & 0.01270 & 1.01479 & 0.01678 & 0.01868 & 0.02047 \\ 
        0.00139 & 0.00414 & 0.00686 & 0.00950 & 0.01204 & 0.01447 & 0.01678 & 1.01896 & 0.02100 & 0.02291 \\ 
        0.00157 & 0.00469 & 0.00775 & 0.01070 & 0.01352 & 0.01618 & 0.01868 & 0.02100 & 1.02315 & 0.02513 \\ 
        0.00176 & 0.00524 & 0.00863 & 0.01188 & 0.01495 & 0.01782 & 0.02047 & 0.02291 & 0.02513 & 1.02717 
    \end{pmatrix}
\end{equation}

\newcommand\scalemath[2]{\scalebox{#1}{\mbox{\ensuremath{\displaystyle #2}}}}

\begin{equation}
    \scalemath{0.8}{
        (E + \lambda A)^{-1} = \begin{pmatrix}
            0.99992 & -0.00025 & -0.00041 & -0.00058 & -0.00074 & -0.00091 & -0.00108 & -0.00125 & -0.00142 & -0.00159 \\ 
            -0.00025 & 0.99926 & -0.00124 & -0.00173 & -0.00223 & -0.00273 & -0.00323 & -0.00373 & -0.00423 & -0.00473 \\ 
            -0.00041 & -0.00124 & 0.99794 & -0.00288 & -0.00371 & -0.00453 & -0.00535 & -0.00617 & -0.00698 & -0.00780 \\ 
            -0.00058 & -0.00173 & -0.00288 & 0.99597 & -0.00517 & -0.00631 & -0.00743 & -0.00854 & -0.00964 & -0.01072 \\ 
            -0.00074 & -0.00223 & -0.00371 & -0.00517 & 0.99338 & -0.00805 & -0.00946 & -0.01083 & -0.01218 & -0.01349 \\ 
            -0.00091 & -0.00273 & -0.00453 & -0.00631 & -0.00805 & 0.99024 & -0.01142 & -0.01302 & -0.01457 & -0.01606 \\ 
            -0.00108 & -0.00323 & -0.00535 & -0.00743 & -0.00946 & -0.01142 & 0.98670 & -0.01510 & -0.01681 & -0.01843 \\ 
            -0.00125 & -0.00373 & -0.00617 & -0.00854 & -0.01083 & -0.01302 & -0.01510 & 0.98295 & -0.01889 & -0.02060 \\ 
            -0.00142 & -0.00423 & -0.00698 & -0.00964 & -0.01218 & -0.01457 & -0.01681 & -0.01889 & 0.97919 & -0.02258 \\ 
            -0.00159 & -0.00473 & -0.00780 & -0.01072 & -0.01349 & -0.01606 & -0.01843 & -0.02060 & -0.02258 & 0.97562 
        \end{pmatrix}
    }
\end{equation}

\begin{spacing}{1.5}
Следовательно, \\
$\lVert E + \lambda A \rVert = 1.1559365$, \\
$\lVert (E + \lambda A)^{-1} \rVert = 1.09161556563663$ \\
и число обусловленности \\
cond($E + \lambda A$) = $\lVert E + \lambda A \rVert \cdot \lVert (E + \lambda A)^{-1} \rVert = 1.26183827628753$. \\
Таким образом, матрица СЛАУ (17) хорошо обусловлена.\\
\end{spacing}

Найдем решение ${^>x} = (E + \lambda A)^{-1} {^>b}$ СЛАУ.\\

\begin{equation}
    \scalemath{0.8}{
        {^>b} = \begin{pmatrix}
            1.00009 & 0.00028 & 0.00046 & 0.00065 & 0.00083 & 0.00102 & 0.00120 & 0.00139 & 0.00157 & 0.00176 \\ 
            0.00028 & 1.00083 & 0.00139 & 0.00194 & 0.00249 & 0.00305 & 0.00360 & 0.00414 & 0.00469 & 0.00524 \\ 
            0.00046 & 0.00139 & 1.00231 & 0.00323 & 0.00414 & 0.00505 & 0.00596 & 0.00686 & 0.00775 & 0.00863 \\ 
            0.00065 & 0.00194 & 0.00323 & 1.00451 & 0.00578 & 0.00704 & 0.00828 & 0.00950 & 0.01070 & 0.01188 \\ 
            0.00083 & 0.00249 & 0.00414 & 0.00578 & 1.00739 & 0.00898 & 0.01053 & 0.01204 & 0.01352 & 0.01495 \\ 
            0.00102 & 0.00305 & 0.00505 & 0.00704 & 0.00898 & 1.01087 & 0.01270 & 0.01447 & 0.01618 & 0.01782 \\ 
            0.00120 & 0.00360 & 0.00596 & 0.00828 & 0.01053 & 0.01270 & 1.01479 & 0.01678 & 0.01868 & 0.02047 \\ 
            0.00139 & 0.00414 & 0.00686 & 0.00950 & 0.01204 & 0.01447 & 0.01678 & 1.01896 & 0.02100 & 0.02291 \\ 
            0.00157 & 0.00469 & 0.00775 & 0.01070 & 0.01352 & 0.01618 & 0.01868 & 0.02100 & 1.02315 & 0.02513 \\ 
            0.00176 & 0.00524 & 0.00863 & 0.01188 & 0.01495 & 0.01782 & 0.02047 & 0.02291 & 0.02513 & 1.02717 
        \end{pmatrix}
        \begin{pmatrix}
            1 \\ 
            1 \\ 
            1 \\ 
            1 \\ 
            1 \\ 
            1 \\ 
            1 \\ 
            1 \\ 
            1 \\ 
            1 
        \end{pmatrix}
        =
        \begin{pmatrix}
            1.00925000000000\\ 
            1.02764640000000\\ 
            1.04578010000000\\ 
            1.06349940000000\\ 
            1.08065630000000\\ 
            1.09716940000000\\ 
            1.11299060000000\\ 
            1.12804960000000\\ 
            1.14236860000000\\ 
            1.15593650000000\\ 
        \end{pmatrix}
    }
\end{equation}

\begin{equation*}
    \scalemath{0.7}{
        {^>x} = \begin{pmatrix}
            0.99992 & -0.00025 & -0.00041 & -0.00058 & -0.00074 & -0.00091 & -0.00108 & -0.00125 & -0.00142 & -0.00159 \\ 
            -0.00025 & 0.99926 & -0.00124 & -0.00173 & -0.00223 & -0.00273 & -0.00323 & -0.00373 & -0.00423 & -0.00473 \\ 
            -0.00041 & -0.00124 & 0.99794 & -0.00288 & -0.00371 & -0.00453 & -0.00535 & -0.00617 & -0.00698 & -0.00780 \\ 
            -0.00058 & -0.00173 & -0.00288 & 0.99597 & -0.00517 & -0.00631 & -0.00743 & -0.00854 & -0.00964 & -0.01072 \\ 
            -0.00074 & -0.00223 & -0.00371 & -0.00517 & 0.99338 & -0.00805 & -0.00946 & -0.01083 & -0.01218 & -0.01349 \\ 
            -0.00091 & -0.00273 & -0.00453 & -0.00631 & -0.00805 & 0.99024 & -0.01142 & -0.01302 & -0.01457 & -0.01606 \\ 
            -0.00108 & -0.00323 & -0.00535 & -0.00743 & -0.00946 & -0.01142 & 0.98670 & -0.01510 & -0.01681 & -0.01843 \\ 
            -0.00125 & -0.00373 & -0.00617 & -0.00854 & -0.01083 & -0.01302 & -0.01510 & 0.98295 & -0.01889 & -0.02060 \\ 
            -0.00142 & -0.00423 & -0.00698 & -0.00964 & -0.01218 & -0.01457 & -0.01681 & -0.01889 & 0.97919 & -0.02258 \\ 
            -0.00159 & -0.00473 & -0.00780 & -0.01072 & -0.01349 & -0.01606 & -0.01843 & -0.02060 & -0.02258 & 0.97562 
        \end{pmatrix}
        \begin{pmatrix}
            1.00925000000000\\ 
            1.02764640000000\\ 
            1.04578010000000\\ 
            1.06349940000000\\ 
            1.08065630000000\\ 
            1.09716940000000\\ 
            1.11299060000000\\ 
            1.12804960000000\\ 
            1.14236860000000\\ 
            1.15593650000000\\ 
        \end{pmatrix}
        =
    }
\end{equation*}

\begin{equation}
    = [ 1,1,1,1,1,1,1,1,1,1 \rangle
\end{equation}

Погрешность решения приближенной СЛАУ найдем так:

\begin{equation}
    {^> \Delta x} = (E + \lambda A)^{-1} {^> \Delta b}, \textrm{ где}
\end{equation}

$
    {^> \Delta b} = 
    [ \Delta b^1, \dots, \Delta b^{10} \rangle = 
    0.01 \cdot [ b^1, -b^2, b^3, -b^4, b^5, -b^6, b^7, -b^8, b^9, -b^{10} \rangle \in {^> \mathbb{R} ^{10}} = 
$ \\

\begin{equation*}
    = [ 
    0.0100925000000000,
    -0.0102764640000000,
    0.0104578010000000,
\end{equation*}
\vspace{-1\baselineskip}
\begin{equation*}
    -0.0106349940000000,
    0.0108065630000000,
    -0.0109716940000000,
\end{equation*}
\vspace{-1\baselineskip}
\begin{equation*}
    0.0111299060000000,
    -0.0112804960000000,
    0.0114236860000000,
\end{equation*}
\vspace{-1\baselineskip}
\begin{equation}
    -0.0115593650000000 
    \rangle
\end{equation}

\begin{equation*}
    \scalemath{0.7}{
        {^> \Delta x} = \begin{pmatrix}
            0.99992 & -0.00025 & -0.00041 & -0.00058 & -0.00074 & -0.00091 & -0.00108 & -0.00125 & -0.00142 & -0.00159 \\ 
            -0.00025 & 0.99926 & -0.00124 & -0.00173 & -0.00223 & -0.00273 & -0.00323 & -0.00373 & -0.00423 & -0.00473 \\ 
            -0.00041 & -0.00124 & 0.99794 & -0.00288 & -0.00371 & -0.00453 & -0.00535 & -0.00617 & -0.00698 & -0.00780 \\ 
            -0.00058 & -0.00173 & -0.00288 & 0.99597 & -0.00517 & -0.00631 & -0.00743 & -0.00854 & -0.00964 & -0.01072 \\ 
            -0.00074 & -0.00223 & -0.00371 & -0.00517 & 0.99338 & -0.00805 & -0.00946 & -0.01083 & -0.01218 & -0.01349 \\ 
            -0.00091 & -0.00273 & -0.00453 & -0.00631 & -0.00805 & 0.99024 & -0.01142 & -0.01302 & -0.01457 & -0.01606 \\ 
            -0.00108 & -0.00323 & -0.00535 & -0.00743 & -0.00946 & -0.01142 & 0.98670 & -0.01510 & -0.01681 & -0.01843 \\ 
            -0.00125 & -0.00373 & -0.00617 & -0.00854 & -0.01083 & -0.01302 & -0.01510 & 0.98295 & -0.01889 & -0.02060 \\ 
            -0.00142 & -0.00423 & -0.00698 & -0.00964 & -0.01218 & -0.01457 & -0.01681 & -0.01889 & 0.97919 & -0.02258 \\ 
            -0.00159 & -0.00473 & -0.00780 & -0.01072 & -0.01349 & -0.01606 & -0.01843 & -0.02060 & -0.02258 & 0.97562 
        \end{pmatrix}
        \begin{pmatrix}
            0.0100925000000000  \\
            -0.0102764640000000 \\ 
            0.0104578010000000  \\
            -0.0106349940000000 \\ 
            0.0108065630000000  \\
            -0.0109716940000000 \\ 
            0.0111299060000000  \\
            -0.0112804960000000 \\ 
            0.0114236860000000  \\
            -0.0115593650000000 \\
        \end{pmatrix}
        =
    }
\end{equation*}

\begin{equation}
     = \begin{pmatrix}
        0.0101022344981114\\ 
        -0.0102475051573947\\ 
        0.0105054578609862\\ 
        -0.0105695306103990\\ 
        0.0108887038899682\\ 
        -0.0108740881677665\\ 
        0.0112415673850983\\ 
        -0.0111558957774656\\ 
        0.0115598593276807\\ 
        -0.0114126855030443
    \end{pmatrix}
\end{equation}

Найдем решение приближенной СЛАУ:

\begin{equation}
    {^> x} + {^> \Delta x} = \begin{pmatrix}
        1.01010223449811\\ 
        0.989752494842605\\ 
        1.01050545786099\\ 
        0.989430469389601\\ 
        1.01088870388997\\ 
        0.989125911832234\\ 
        1.01124156738510\\ 
        0.988844104222534\\ 
        1.01155985932768\\ 
        0.988587314496956\\
   \end{pmatrix}
\end{equation}

Рассмотрим результаты: \\\\
$ \lVert {^>x}         \rVert = 1 $, \\\\
$ \lVert {^> \Delta x} \rVert = 0.0115598593276807 $, \\\\
$ \lVert {^>b}         \rVert = 1.1559365 $ , \\\\
$ \lVert {^> \Delta b} \rVert = 0.011423686 $ , \\\\
$\frac{\lVert {^> \Delta x} \rVert}{\lVert {^>x} \rVert} = 0.0115598593276807 $ , \\\\
cond($A$) $\frac{\lVert {^> \Delta b} \rVert}{\lVert {^>b} \rVert} = 0.0124702734545453 $ , \\\\
т.к. $0.0115598593276807 < 0.0124702734545453$, \\\\
можем сделать вывод, что неравенство $\frac{\lVert {^> \Delta x} \rVert}{\lVert {^>x} \rVert} < \textrm{cond}(A) \frac{\lVert {^> \Delta b} \rVert}{\lVert {^>b} \rVert}$ - выполняется \\

\textbf{Найдем решение СЛАУ, которая получается из СЛАУ (17) делением каждого ее $i$-го уравнения $(i = \overline{1, 10})$ га число $b_i + \Delta b_i$.}

Получим матрицу

\begin{equation}
    B = \begin{pmatrix}
        0.98112 & 0.00027 & 0.00045 & 0.00064 & 0.00082 & 0.0010 & 0.00118 & 0.00136 & 0.00154 & 0.00172 \\ 
        0.00027 & 0.98374 & 0.00136 & 0.00191 & 0.00245 & 0.00299 & 0.00353 & 0.00407 & 0.00461 & 0.00515 \\ 
        0.00044 & 0.00131 & 0.94894 & 0.00306 & 0.00392 & 0.00479 & 0.00564 & 0.00649 & 0.00734 & 0.00817 \\ 
        0.00061 & 0.00184 & 0.00307 & 0.95407 & 0.00549 & 0.00668 & 0.00786 & 0.00902 & 0.01016 & 0.01128 \\ 
        0.00076 & 0.00228 & 0.00380 & 0.00530 & 0.92297 & 0.00822 & 0.00965 & 0.01103 & 0.01238 & 0.01370 \\ 
        0.00094 & 0.00280 & 0.00465 & 0.00648 & 0.00826 & 0.93065 & 0.01169 & 0.01333 & 0.01490 & 0.01640 \\ 
        0.00107 & 0.00320 & 0.00530 & 0.00736 & 0.00937 & 0.01130 & 0.90274 & 0.01493 & 0.01662 & 0.01821 \\ 
        0.00124 & 0.00371 & 0.00614 & 0.00850 & 0.01078 & 0.01296 & 0.01503 & 0.91242 & 0.01880 & 0.02051 \\ 
        0.00136 & 0.00407 & 0.00672 & 0.00927 & 0.01171 & 0.01402 & 0.01619 & 0.01820 & 0.88677 & 0.02178 \\ 
        0.00154 & 0.00457 & 0.00754 & 0.01038 & 0.01306 & 0.01557 & 0.01789 & 0.02002 & 0.02196 & 0.89758
   \end{pmatrix}
\end{equation}

\begin{equation}
    \scalemath{0.8}{
        B^{-1} = \begin{pmatrix}
            1.01926 & -0.00025 & -0.00044 & -0.00061 & -0.00081 & -0.00099 & -0.00121 & -0.00139 & -0.00164 & -0.00182 \\ 
            -0.00025 & 1.01662 & -0.00131 & -0.00183 & -0.00243 & -0.00296 & -0.00363 & -0.00416 & -0.00488 & -0.00541 \\ 
            -0.00042 & -0.00126 & 1.05406 & -0.00304 & -0.00404 & -0.00492 & -0.00601 & -0.00689 & -0.00806 & -0.00892 \\ 
            -0.00059 & -0.00176 & -0.00305 & 1.04862 & -0.00565 & -0.00685 & -0.00835 & -0.00954 & -0.01113 & -0.01227 \\ 
            -0.00076 & -0.00227 & -0.00391 & -0.00545 & 1.08423 & -0.00875 & -0.01063 & -0.01210 & -0.01405 & -0.01543 \\ 
            -0.00093 & -0.00277 & -0.00478 & -0.00664 & -0.00879 & 1.07560 & -0.01283 & -0.01454 & -0.01681 & -0.01838 \\ 
            -0.00110 & -0.00328 & -0.00565 & -0.00782 & -0.01032 & -0.01240 & 1.10917 & -0.01686 & -0.01939 & -0.02109 \\ 
            -0.00127 & -0.00379 & -0.00651 & -0.00900 & -0.01182 & -0.01414 & -0.01697 & 1.09772 & -0.02179 & -0.02358 \\ 
            -0.00145 & -0.00430 & -0.00738 & -0.01015 & -0.01329 & -0.01582 & -0.01890 & -0.02109 & 1.12978 & -0.02584 \\ 
            -0.00162 & -0.00481 & -0.00823 & -0.01129 & -0.01472 & -0.01744 & -0.02072 & -0.02301 & -0.02605 & 1.11648
    \end{pmatrix}
    }
\end{equation}

Таким образом, получим СЛАУ: $B \cdot {^> x'} = [ 1,1,1,1,1,1,1,1,1,1 \rangle = b_i + \Delta b_i$ \\
Отсюда получаем:

\begin{equation}
    {^> x'} = B^{-1} \cdot [ 1,1,1,1,1,1,1,1,1,1 \rangle = \begin{pmatrix}
        1.01010223449811 \\ 
        0.989752494842605 \\ 
        1.01050545786099 \\ 
        0.989430469389601 \\ 
        1.01088870388997 \\ 
        0.989125911832234 \\ 
        1.01124156738510 \\ 
        0.988844104222535 \\ 
        1.01155985932768 \\ 
        0.988587314496955 \\
   \end{pmatrix}
\end{equation}

\begin{equation}
    \scalemath{0.8}{
        {^>b'} = \begin{pmatrix}
            0.98112 & 0.00027 & 0.00045 & 0.00064 & 0.00082 & 0.0010 & 0.00118 & 0.00136 & 0.00154 & 0.00172 \\ 
            0.00027 & 0.98374 & 0.00136 & 0.00191 & 0.00245 & 0.00299 & 0.00353 & 0.00407 & 0.00461 & 0.00515 \\ 
            0.00044 & 0.00131 & 0.94894 & 0.00306 & 0.00392 & 0.00479 & 0.00564 & 0.00649 & 0.00734 & 0.00817 \\ 
            0.00061 & 0.00184 & 0.00307 & 0.95407 & 0.00549 & 0.00668 & 0.00786 & 0.00902 & 0.01016 & 0.01128 \\ 
            0.00076 & 0.00228 & 0.00380 & 0.00530 & 0.92297 & 0.00822 & 0.00965 & 0.01103 & 0.01238 & 0.01370 \\ 
            0.00094 & 0.00280 & 0.00465 & 0.00648 & 0.00826 & 0.93065 & 0.01169 & 0.01333 & 0.01490 & 0.01640 \\ 
            0.00107 & 0.00320 & 0.00530 & 0.00736 & 0.00937 & 0.01130 & 0.90274 & 0.01493 & 0.01662 & 0.01821 \\ 
            0.00124 & 0.00371 & 0.00614 & 0.00850 & 0.01078 & 0.01296 & 0.01503 & 0.91242 & 0.01880 & 0.02051 \\ 
            0.00136 & 0.00407 & 0.00672 & 0.00927 & 0.01171 & 0.01402 & 0.01619 & 0.01820 & 0.88677 & 0.02178 \\ 
            0.00154 & 0.00457 & 0.00754 & 0.01038 & 0.01306 & 0.01557 & 0.01789 & 0.02002 & 0.02196 & 0.89758
        \end{pmatrix}
        \begin{pmatrix}
            1 \\ 
            1 \\ 
            1 \\ 
            1 \\ 
            1 \\ 
            1 \\ 
            1 \\ 
            1 \\ 
            1 \\ 
            1 
        \end{pmatrix}
        =
        \begin{pmatrix}
            0.990099009900990 \\ 
            1.01010101010101 \\ 
            0.990099009900990 \\ 
            1.01010101010101 \\ 
            0.990099009900990 \\ 
            1.01010101010101 \\ 
            0.990099009900990 \\ 
            1.01010101010101 \\ 
            0.990099009900990 \\ 
            1.01010101010101 \\ 
        \end{pmatrix}
    }
\end{equation}

$
    {^> \Delta b'} = 
    [ \Delta b'^1, \dots, \Delta b'^{10} \rangle = 
    0.01 \cdot [ b'^1, -b'^2, b'^3, -b'^4, b'^5, -b'^6, b'^7, -b'^8, b'^9, -b'^{10} \rangle \in {^> \mathbb{R} ^{10}} = 
$ 

\begin{equation*}
    = [ 
    0.00990099009900990,
    -0.0101010101010101,
    0.00990099009900990
\end{equation*}
\vspace{-1\baselineskip}
\begin{equation*}
    -0.0101010101010101,
    0.00990099009900990,
    -0.0101010101010101,
\end{equation*}
\vspace{-1\baselineskip}
\begin{equation*}
    0.00990099009900990,
    -0.0101010101010101,
    0.00990099009900990,
\end{equation*}
\vspace{-1\baselineskip}
\begin{equation}
    -0.0101010101010101
    \rangle
\end{equation}

\begin{equation}
    {^> \Delta x'} = B^{-1} \cdot {^> \Delta b'} = 
    \begin{pmatrix}
        0.0101022344981114 \\ 
        -0.0102475051573948 \\ 
        0.0105054578609862 \\ 
        -0.0105695306103990 \\ 
        0.0108887038899682 \\ 
        -0.0108740881677665 \\ 
        0.0112415673850983 \\ 
        -0.0111558957774656 \\ 
        0.0115598593276807 \\ 
        -0.0114126855030442 
    \end{pmatrix}
\end{equation}

\begin{spacing}{1.5}
Таким образом, абсолютная погрешности: \\
$\lVert {^> \Delta x'} \rVert = 0.0115598593276807 $ и \\ 
$\lVert {^> \Delta x} \rVert = 0.0115598593276807 $ \\ 
решений приближенной и исходной СЛАУ совпадают.
\end{spacing}

\section*{\centering{ \large Результаты}}

\begin{spacing}{1.5}
Число обусловленности матрицы СЛАУ (17) cond($E + \lambda A$) $= 1.26183827628753 < 10$, то есть матрица $E + \lambda A$ хорошо обусловлена. 
Следствие этого является малая относительная погрешность при решении приближенной СЛАУ $\frac{\lVert {^> \Delta x} \rVert}{\lVert {^>x} \rVert} = 0.0115598593276807$.
Эта погрешность в 1.0787565056855462 раз меньше верхней границы относительной погрешности $\textrm{cond}(A) \frac{\lVert {^> \Delta b} \rVert}{\lVert {^>b} \rVert} = 0.0124702734545453$.
Кроме того, при делении каждого $i$-го уравнения ($i = \overline{1, 10}$) СЛАУ на число $b_i + \Delta b_i$ абсолютная погрешность не изменилась: $\lVert {^> \Delta x'} \rVert = \lVert {^> \Delta x} \rVert = 0.0115598593276807 $.
\end{spacing}

\end{document}
