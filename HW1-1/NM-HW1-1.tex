\documentclass[a4paper, 12pt]{extarticle}

\usepackage[utf8]{inputenc}
\usepackage[paper=a4paper, left=10mm, right=10mm, top=20mm, bottom=20mm]{geometry}
\usepackage[russian]{babel}
\usepackage{amsmath}
\usepackage[T2A]{fontenc}
\usepackage{textcomp}
\usepackage{parskip}

\linespread{1}

\begin{document}

% титульник

\pagestyle{empty}

\section*{\centering{ \large МИНИСТЕРСТВО ОБРАЗОВАНИЯ И НАУКИ РОССИЙСКОЙ ФЕДЕРАЦИИ ГОСУДАРСТВЕННОЕ БЮДЖЕТНОЕ ОБРАЗОВАТЕЛЬНОЕ УЧРЕЖДЕНИЕ ВЫСШЕГО ПРОФЕССИОНАЛЬНОГО ОБРАЗОВАНИЯ}}

\vspace{3\baselineskip}

\section*{\centering{ \large «Московский государственный технический университет имени Н.Э. Баумана» (МГТУ им. Н.Э. Баумана)}}

\vspace{3\baselineskip}

\section*{\centering{ \large ФАКУЛЬТЕТ ФУНДАМЕНТАЛЬНЫЕ НАУКИ }}
\vspace{-1\baselineskip}
\section*{\centering{ \large КАФЕДРА «ВЫЧИСЛИТЕЛЬНАЯ МАТЕМАТИКА И МАТЕМАТИЧЕСКАЯ ФИЗИКА»}}

\vspace{1\baselineskip}

\section*{\centering \normalsize Направление: \fontseries{bx}\selectfont{Математика и компьютерные науки}}

\vspace{1\baselineskip}

\section*{\centering{ \normalsize Дисциплина: Численные методы}}

\vspace{1\baselineskip}

\section*{\centering{ \normalsize Домашняя работа \textnumero 1}}
\vspace{-1\baselineskip}
\section*{\centering{ \normalsize «Погрешности при решении СЛАУ»}}
\vspace{-1\baselineskip}
\section*{\centering{ \normalsize Группа ФН11-52Б}}

\vspace{1\baselineskip}

\section*{\centering{ \normalsize Вариант 9}}

\vspace{1\baselineskip}

\begin{flushright}
    { \normalsize \textbf{Студент: Очкин Н. В. }}
\end{flushright}

\vspace{1\baselineskip}

\begin{flushright}
    { \normalsize \textbf{Преподаватель: Кутыркин В. А.}}
\end{flushright}

\vspace{1\baselineskip}

\begin{flushright}
    { \normalsize \textbf{Оценка: } \hspace{2.5cm} }
\end{flushright}

\vspace{2\baselineskip}

\section*{\centering{ \normalsize Москва 2024}}

\newpage

% Задание 1.1

\pagestyle{plain}
\setcounter{page}{1}

\section*{\centering{ \large Задание 1.1}}

Дана СЛАУ ( $N$ – номер студента в журнале, $\alpha = \frac{n - 50}{100}$, где $n$ – номер группы):

\begin{equation}
    \begin{cases}
        150(1 + 0.5 N + \alpha)x^1 + 150(1 + 0.5 N)x^2 + 150(1 + 0.5N)x^3 = 150(3 + 1.5N + \alpha);\\
        150.1 \cdot (1 + 0.5 N)x^1 + 149.9 \cdot (1 + 0.5 N + \alpha)x^2 + 150(1 + 0.5N)x^3 = 150(3 + 1.5 N + \alpha);\\
        149.9 \cdot (1 + 0.5N)x^1 + 150 \cdot (1 + 0.5 N)x^2 + 150.1 \cdot (1 + 0.5 N + \alpha)x^3 = 150 (3 + 1.5 N + \alpha);
    \end{cases}
\end{equation}

Предполагается, что ошибка в матрице этой СЛАУ достаточно мала и относительная
ошибка в её правой части равна 0,01. Приближённая СЛАУ имеет вид:

\begin{equation}
    \begin{cases}
        150(1 + 0.5 N + \alpha)x^1 + 150(1 + 0.5 N)x^2 + 150(1 + 0.5N)x^3 = 150(3 + 1.5N + \alpha)(1 + 0.01);\\
        150.1 \cdot (1 + 0.5 N)x^1 + 149.9 \cdot (1 + 0.5 N + \alpha)x^2 + 150(1 + 0.5N)x^3 = 150(3 + 1.5 N + \alpha)(1 - 0.01);\\
        149.9 \cdot (1 + 0.5N)x^1 + 150 \cdot (1 + 0.5 N)x^2 + 150.1 \cdot (1 + 0.5 N + \alpha)x^3 = 150 (3 + 1.5 N + \alpha)(1 + 0.01);
    \end{cases}
\end{equation}

Требуется найти число обусловленности матрицы рассматриваемой СЛАУ и
относительную погрешность в решении приближённой СЛАУ. Затем, прокомментировать
получившиеся результаты.

\section*{\centering{ \large Решение}}

Подставим $N = 9$ и $\alpha = \frac{52 - 50}{100} = 0.02$ в (1):

\begin{equation}
    \begin{cases}
        828x^1 + 825x^2 + 825x^3 = 2478;\\
        825.55x^1 + 827.448x^2 + 825x^3 = 2478;\\
        824.45x^1 + 825x^2 + 828.552x^3 = 2478;
    \end{cases}
\end{equation}

Матрица СЛАУ (1) имеет вид:

\begin{equation}
    A = \begin{pmatrix}
        828    & 825     & 825 \\
        825.55 & 827.448 & 825 \\
        824.45 & 825     & 828.552
    \end{pmatrix}
\end{equation}

\textbf{1. Найдем число обусловленности матрицы A.}

Число обусловленности матрицы равно:

\begin{equation}
    \textrm{cond}(A) = \lVert A \rVert \cdot \lVert A^{-1} \rVert
\end{equation}

Вычислим обратную матрицу с помощью функции inv() библиотеки sympy:

\begin{equation}
    A^{-1} = \begin{pmatrix}
        0.230115056512510   & -0.135989231310137 & -0.0937223080651046 \\ 
        -0.178193673016471  & 0.272396398402827  & -0.0937988785782234 \\ 
        -0.0515460443076003 & -0.135912660797018 & 0.187861995036292 
    \end{pmatrix}
\end{equation}

Норма матрицы вычисляется по формуле:
\begin{equation}
    \lVert A \rVert = \textrm{max} \left\{ \sum_{j=1}^{n} |a^i_j|:i = \overline{1, n} \right\}
\end{equation}

% Для этого сначала вычислим матрицы, составленные из модулей элементов исходных
% матриц, воспользовавшись функцией applyfunc(abs):

% \begin{equation}
%     |A| = \begin{pmatrix}
%         828.000000000000 & 825.000000000000 & 825.000000000000 \\ 
%         825.550000000000 & 827.448000000000 & 825.000000000000 \\ 
%         824.450000000000 & 825.000000000000 & 828.552000000000 
%     \end{pmatrix}
% \end{equation}

% \begin{equation}
%     |A^{-1}| = \begin{pmatrix}
%         0.230115056512510 & 0.135989231310137 & 0.0937223080651046 \\ 
%         0.178193673016471 & 0.272396398402827 & 0.0937988785782234 \\ 
%         0.0515460443076003 & 0.135912660797018 & 0.187861995036292 
%     \end{pmatrix}
% \end{equation}

Вычислим нормы, воспользовавшись функцией norm():

\begin{equation}
    \lVert A \rVert      = 2478.00387699616 
\end{equation}

\begin{equation}
    \lVert A^{-1} \rVert = 0.501400308248106
\end{equation}

Теперь вычислим число обусловленности матрицы:

\begin{equation}
    \textrm{cond}(A) = \lVert A \rVert \cdot \lVert A^{-1} \rVert = 2478.00387699616 \cdot 0.501400308248106 \approx 1242.47190776588
\end{equation}

Следовательно, СЛАУ (3) плохо обусловлена.

\end{document}
