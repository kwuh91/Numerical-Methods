\documentclass[letterpaper, 11pt]{extarticle}
% \usepackage{fontspec}

% ==================================================

% document parameters
% \usepackage[spanish, mexico, es-lcroman]{babel}
\usepackage[utf8]{inputenc}
\usepackage[english, russian]{babel}
\usepackage[margin = 1in]{geometry}
\usepackage{setspace}
% \usepackage[paper=a4paper, left=15mm, right=15mm, top=20mm, bottom=20mm]{geometry}

% ==================================================

% Packages for math
\usepackage{mathrsfs}
\usepackage{amsfonts}
\usepackage{amsmath}
\usepackage{amsthm}
\usepackage{amssymb}
\usepackage{physics}
\usepackage{dsfont}
\usepackage{esint}
\usepackage{mathtools}

% ==================================================

% Packages for writing
\usepackage{enumerate}
\usepackage[shortlabels]{enumitem}
\usepackage{framed}
\usepackage{csquotes}

% ==================================================

% Miscellaneous packages
\usepackage{float}
\usepackage{tabularx}
\usepackage{xcolor}
\usepackage{multicol}
\usepackage{subcaption}
\usepackage{caption}
\usepackage{fancyhdr}
\usepackage{graphicx}
\captionsetup{format = hang, margin = 10pt, font = small, labelfont = bf}

% Citation
\usepackage[round, authoryear]{natbib}

% Hyperlinks setup
\usepackage{hyperref}
\definecolor{links}{rgb}{0.36,0.54,0.66}
\hypersetup{
   colorlinks = true,
    linkcolor = black,
     urlcolor = blue,
    citecolor = blue,
    filecolor = blue,
    pdfauthor = {Author},
     pdftitle = {Title},
   pdfsubject = {subject},
  pdfkeywords = {one, two},
  pdfproducer = {LaTeX},
   pdfcreator = {pdfLaTeX},
}


\pagestyle{empty}

\begin{document}

\begin{singlespace}
\setstretch{1}

\parbox{\textwidth}{
    \centering{
        МИНИСТЕРСТВО ОБРАЗОВАНИЯ И НАУКИ РОССИЙСКОЙ ФЕДЕРАЦИИ\\
        ГОСУДАРСТВЕННОЕ БЮДЖЕТНОЕ ОБРАЗОВАТЕЛЬНОЕ УЧРЕЖДЕНИЕ\\
        ВЫСШЕГО ПРОФЕССИОНАЛЬНОГО ОБРАЗОВАНИЯ
    }
}

\vspace{5em}

\parbox{\textwidth}{
    \centering{
        \guillemotleft Московский государственный технический\\
        университет имени Н.Э. Баумана\guillemotright\\
        (МГТУ им. Н.Э. Баумана)}
}

\vspace{5em}

\parbox{\textwidth}{
    \centering{
        ФАКУЛЬТЕТ ФУНДАМЕНТАЛЬНЫХ НАУК\\
        КАФЕДРА\\
        \guillemotleft ВЫЧИСЛИТЕЛЬНАЯ МАТЕМАТИКА И МАТЕМАТИЧЕСКАЯ ФИЗИКА\guillemotright
    }
}

\vspace{3em}

\parbox{\textwidth}{
    \centering{
        Направление: \textbf{Математика и компьютерные науки}
    }
}

\vspace{3em}

\parbox{\textwidth}{
    \centering{
        Дисциплина: Численные методы
    }
}

\vspace{3em}

\parbox{\textwidth}{
    \centering{
        Домашняя работа №1.2\\
        \guillemotleft Метод наименьших квадратов и модели регрессии\guillemotright\\
        Группа ФН11-52Б
    }
}

\vspace{3em}

\parbox{\textwidth}{
    \centering{
        Вариант №9
    }
}

\vspace{3em}

\parbox{\textwidth}{
    \raggedleft{
        Студент: Очкин Н.В.
    }
}

\vspace{3em}

\parbox{\textwidth}{
    \raggedleft{
        Преподаватель:  Кутыркин В.А.
    }
}

\vspace{3em}

\parbox{\textwidth}{
    \centering{
        Оценка:
    }
}

\vspace{7em}

\parbox{\textwidth}{
    \centering{
        Москва, 2024
    }
}

\end{singlespace}

\newpage

\doublespacing

\section*{\centering{Задание 2.1}}

Дана модель линейной регрессии:

\begin{equation}
	Y = x^0_* + z_1x^1_* + z_2x^2_* + z_3x^3_* + z_4x^4_* + z_5x^5_* + z_6x^6_* + \varepsilon
\end{equation}

Для оценки неизвестных вектора тренда 
$\prescript{>}{}{x_*} = \left[ x^0_*, x^1_*, \dots, x^k_* \right> \in \prescript{>}{}{\mathbb{E}^{k+1}}$
и параметра $\sigma$ от случайной составляющей $\varepsilon \sim \mathcal{N}(0, \sigma)$ модели линейной регрессии (1). 
проводился эксперимент, в котором получены $m = 20$ значений $y^1, \dots, y^m \in \mathbb{R}$
регрессора модели (1) для $m$ различных наборов 
$\prescript{<}{}{z^1} = \left< z^1_1, \dots,  z^1_6\right], \dots, \prescript{<}{}{z^m} = \left< z^m_1, \dots,  z^m_6\right] 
\in \in \prescript{<}{}{\mathbb{R}^{6}}$ шести факторов модели (1).

Требуется получить оценки вектора тренда
$\prescript{>}{}{x_*} = \left[ x^0_*, x^1_*, \dots, x^k_* \right> \in \prescript{>}{}{\mathbb{E}^{k+1}}$
и параметра $\sigma$ от случайной составляющей $\varepsilon \sim \mathcal{N}(0, \sigma)$ модели линейной регрессии (1). 
Если возможно, редуцировать модель регрессии (1) до приведённой модели. 
Результаты расчётов проиллюстрировать графически, сопроводив их необходимыми комментариями.

\vspace{-0.75\baselineskip}

\section*{\centering{ \large Решение}}

\vspace{-0.75\baselineskip}

\begin{equation*}
    \textrm{N} = 9, \alpha = -0.025
\end{equation*}

\renewcommand{\arraystretch}{0.65}
\begin{center}
    \begin{tabular}{ | c | c | c | c | c | c | c | c | } 
      \hline
      $\textrm{z}^1$ & $\textrm{z}^2$ & $\textrm{z}^3$ & $\textrm{z}^4$ & $\textrm{z}^5$ & $\textrm{z}^6$ & $\textrm{y} + \alpha$ & $\textrm{y}$ \\
      \hline
      1.158574 & 1.194067 & 1.745872 & 1.566271 & 1.825556 & 1.942503 & 14.77 & 14.795 \\ 
      \hline 
      1.238868 & 1.913419 & 1.182653 & 1.044649 & 1.304209 & 1.924039 & 13.41 & 13.435 \\ 
      \hline 
      1.564043 & 1.561357 & 1.070589 & 1.778954 & 1.226447 & 1.824122 & 13.84 & 13.865 \\ 
      \hline 
      1.737266 & 1.798975 & 1.952239 & 1.752281 & 1.247871 & 1.54796 & 13.6 & 13.625 \\ 
      \hline 
      1.364544 & 1.03122 & 1.380596 & 1.688101 & 1.987396 & 1.058504 & 13.23 & 13.255 \\ 
      \hline 
      1.535295 & 1.742973 & 1.580401 & 1.063356 & 1.999237 & 1.425459 & 14.88 & 14.905 \\ 
      \hline 
      1.780725 & 1.306711 & 1.972594 & 1.68627 & 1.582629 & 1.767235 & 15.39 & 15.415 \\ 
      \hline 
      1.135044 & 1.139164 & 1.686178 & 1.220069 & 1.034577 & 1.019745 & 9.56 & 9.585 \\ 
      \hline 
      1.246498 & 1.114597 & 1.079653 & 1.333415 & 1.054445 & 1.156743 & 10.37 & 10.395 \\ 
      \hline 
      1.416456 & 1.349223 & 1.68038 & 1.003235 & 1.471908 & 1.095523 & 11.95 & 11.975 \\ 
      \hline 
      1.611866 & 1.972991 & 1.443953 & 1.014008 & 1.91699 & 1.182531 & 14.13 & 14.155 \\ 
      \hline 
      1.520585 & 1.427992 & 1.464156 & 1.011505 & 1.108341 & 1.981536 & 13.83 & 13.855 \\ 
      \hline 
      1.229896 & 1.304392 & 1.852107 & 1.705496 & 1.725639 & 1.21482 & 12.51 & 12.535 \\ 
      \hline 
      1.726829 & 1.866756 & 1.074984 & 1.09888 & 1.983154 & 1.256935 & 14.9 & 14.925 \\ 
      \hline 
      1.77279 & 1.363353 & 1.227454 & 1.076754 & 1.656758 & 1.675253 & 15.31 & 15.335 \\ 
      \hline 
      1.418256 & 1.072481 & 1.123447 & 1.438917 & 1.059481 & 1.080325 & 10.67 & 10.695 \\ 
      \hline 
      1.119724 & 1.947356 & 1.372631 & 1.635578 & 1.94058 & 1.112827 & 12.52 & 12.545 \\ 
      \hline 
      1.728446 & 1.802332 & 1.365001 & 1.184759 & 1.119633 & 1.880032 & 14.18 & 14.205 \\ 
      \hline 
      1.161107 & 1.359294 & 1.956206 & 1.143406 & 1.49144 & 1.688437 & 13.02 & 13.045 \\ 
      \hline 
      1.963561 & 1.271859 & 1.250008 & 1.19367 & 1.466262 & 1.624409 & 15.16 & 15.185 \\ 
      \hline 
    \end{tabular}
\end{center}

Вычисления проведём в \textquote{\texttt{python}} с помощью библиотеки \textquote{\texttt{statsmodels}}.\\
Результаты вычислений:

\renewcommand{\arraystretch}{0.65}
\begin{center}
    \begin{tabular}{ | c | c | c | c | c | c | c | c | } 
      \hline
       & Y-пересечение & $\textrm{z}^1$ & $\textrm{z}^2$ & $\textrm{z}^3$ & $\textrm{z}^4$ & $\textrm{z}^5$ & $\textrm{z}^6$ \\
      \hline
      Коэффициенты & 0.017 & 2.9986 & 0.0062 & -0.001 & 0.0005 & 2.9996 & 3.0001 \\ 
      \hline 
     t-статистика & 1.7929 & 820.6219 & 2.0042 & -0.3436 & 0.1451 & 1091.222 & 1062.342 \\ 
      \hline 
     Нижние 95\% & -0.0035 & 2.9907 & -0.0005 & -0.0075 & -0.0064 & 2.9936 & 2.994 \\ 
      \hline 
     Верхние 95\% & 0.0375 & 3.0065 & 0.0129 & 0.0054 & 0.0073 & 3.0055 & 3.0062 \\ 
      \hline 
    \end{tabular}
\end{center}

\end{document}