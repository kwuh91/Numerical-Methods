\documentclass[a4paper, 14pt]{extarticle}

\usepackage{../latexDependencies/misc/preamble2}

\geometry{a4paper}

% Название дисциплины
\newcommand{\subject}{Численные методы} 

% Тип работы
% lab - для лабораторной работы 
% hw  - для домашней     работы
\newcommand{\task}{hw} 

% Номер работы
\newcommand{\taskNumber}{2-1} 

% Название работы
\newcommand{\taskNameOne}{Интерполяция Лагранжа.} 
\newcommand{\taskNameTwo}{Вычисление интерполяционного полинома} 
\newcommand{\taskNameThree}{Лагранжа.} 

% Имя студента
\newcommand{\studentName}{Очкин Н.В.}

% Имя преподававателя
\newcommand{\teacherName}{Кутыркин В.А.}

% Группа
\newcommand{\group}{ФН11-52Б}

% Вариант
\newcommand{\variant}{9}

\begin{document}

\graphicspath{ {../latexDependencies/images} } 
\input{../latexDependencies/frontmatter/titlepage2}

\newgeometry{left=25mm, right=25mm, top=20mm, bottom=20mm}

\graphicspath{ {../latexDependencies/images/HW2-1} }

% Customize section, subsection, subsubsection and paragraph styles
\titleformat{\section}
  {\normalfont\large\bfseries}{\thesection}{1em}{}

\titleformat{\subsection}
  {\normalfont\normalsize\bfseries}{\thesubsection}{1em}{}

\titleformat{\subsubsection}
  {\normalfont\small\bfseries}{\thesubsubsection}{1em}{}

\titleformat{\paragraph}
  {\small\small\bfseries}{\theparagraph}{1em}{}

% \thispagestyle{empty}

% \null\newpage

% \setcounter{tocdepth}{5}
% \setcounter{secnumdepth}{5}

% \pagenumbering{roman}

% \tableofcontents
% \newpage

% \pagenumbering{arabic}
% \setcounter{page}{1}

\setstretch{1}
\linespread{1.1}

\setlength{\parindent}{0pt}

\fontsize{12pt}{16pt}\selectfont

% \definecolor{myblue}{HTML}{0A88C2}
% \definecolor{myred}{HTML}{FF1B1C}
% \definecolor{mygreen}{HTML}{386641}

% \lstdefinestyle{mystyle}{
%     basicstyle=\ttfamily\footnotesize,
%     keywordstyle=\color{myblue},
%     stringstyle=\color{myred},
%     commentstyle=\color{green!50!black},
%     showstringspaces=false,
%     frame=leftline, 
%     framesep=10pt, 
% }

% % Set the style for Python code
% \lstset{style=mystyle, extendedchars=\true}

% --------------------------------------START--------------------------------------

\section*{Задание}\vspace{-20pt}\rule{\linewidth}{0.1mm}

\begin{equation*}
    \begin{cases}
        a_0 + b_0 (x - x_0) + c_0 (x - x_0)^2 + d_0 (x - x_0)^3, & x \in [x_0, x_1] \\ 
        a_1 + b_1 (x - x_1) + c_1 (x - x_1)^2 + d_1 (x - x_1)^3, & x \in [x_1, x_2] \\ 
        \vdots \\ 
        a_{n-2} + b_{n-2} (x - x_{n-2}) + c_{n-2} (x - x_{n-2})^2 + d_{n-2} (x - x_{n-2})^3, & x \in [x_{n-2}, x_{n-1}] 
    \end{cases}   
\end{equation*}

Таким образом, для записи функции естественного кубического сплайна дефекта 1 вам нужно просто подставить найденные коэффициенты $ a_i, b_i, c_i, d_i $ в 
вышеуказанную формулу для каждого интервала $[x_i, x_{i+1}]$.

\end{document}